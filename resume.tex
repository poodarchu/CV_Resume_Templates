% LaTeX Curriculum Vitae Template
%
% Copyright (C) 2004-2009 Jason Blevins <jrblevin@sdf.lonestar.org>
% http://jblevins.org/projects/cv-template/
%
% You may use use this document as a template to create your own CV
% and you may redistribute the source code freely. No attribution is
% required in any resulting documents. I do ask that you please leave
% this notice and the above URL in the source code if you choose to
% redistribute this file.

\documentclass[letterpaper]{article}

\usepackage{hyperref}
\usepackage{geometry}
\usepackage{setspace}

% Comment the following lines to use the default Computer Modern font
% instead of the Palatino font provided by the mathpazo package.
% Remove the 'osf' bit if you don't like the old style figures.
\usepackage[T1]{fontenc}
\usepackage[sc,osf]{mathpazo}

% Set your name here
\def\name{Benjin Zhu}

% Replace this with a link to your CV if you like, or set it empty
% (as in \def\footerlink{}) to remove the link in the footer:
\def\footerlink{http://jblevins.org/projects/cv-template/}

% The following metadata will show up in the PDF properties
\hypersetup{
  colorlinks = true,
  urlcolor = black,
  pdfauthor = {\name},
  pdfkeywords = {computer science, deep learning, statistics, mathematics},
  pdftitle = {\name: Curriculum Vitae},
  pdfsubject = {Curriculum Vitae},
  pdfpagemode = UseNone
}

\geometry{
  body={6.5in, 8.5in},
  left=1.0in,
  top=1.25in
}

% Customize page headers
\pagestyle{headings}
\markright{\name}
\thispagestyle{empty}

% Custom section fonts
\usepackage{sectsty}
\sectionfont{\rmfamily\mdseries\Large}
\subsectionfont{\rmfamily\mdseries\itshape\large}

% Other possible font commands include:
% \ttfamily for teletype,
% \sffamily for sans serif,
% \bfseries for bold,
% \scshape for small caps,
% \normalsize, \large, \Large, \LARGE sizes.

% Don't indent paragraphs.
\setlength\parindent{0em}

% Make lists without bullets
\renewenvironment{itemize}{
  \begin{list}{}{
    \setlength{\leftmargin}{1.5em}
  }
}{
  \end{list}
}

\begin{document}

% Place name at left
%{\huge \name}
% Alternatively, print name centered and bold:
\centerline{\huge \bf \name}

%\vspace{0.25in}

%\begin{minipage}{0.45\linewidth}
%  \href{http://www.unc.edu/}{University of North Carolina} \\
%  Department of Statistics \\
%  Smith Building \\
%  Chapel Hill, NC 27599
%\end{minipage}
\begin{center}
\begin{tabular}{lll}
Phone: (86) 185-2015-9000 & Email: \href{mailto:poodarchu@gmail.com}{\tt poodarchu@gmail.com} & Homepage: \href{https://pdc.one/}{\tt https://www.pdc.one/} \\
  \end{tabular}
\end{center}

\section*{Research Interests}
3D Computer Vision in Robotics


\section*{Education}
\begin{itemize}
  \item \textbf{South China University of Technology} \hfill GuangZhou, China 
        \\ B.Eng. in Software Engineering(\textbf{Excellence Engineer Programme})\hfill Jul 2018
        \\ First Class Honors with GPA 3.6/4.0
\end{itemize}


\section*{Research Experience}

\begin{itemize}
	\item \textbf{Megvii Research} \hfill Beijing, China
          \\  Researcher \hfill Feb 2019--Present
          \vspace{-0.1cm}
		  \begin{itemize}
		    \itemsep0em 
			\item 1. Build MEGVII's 3D Object Detection Codebase; Winner of the nuScenes 3D Object Detection Challenge in WAD, CVPR 2019(First participant); Propose new 3D object detection method ViP(First author)
			\item 2. Research on Key-point based 2D Object Detection, End-to-End NMS, Free Object Detection; Participate in COCO Detection Challenge in ICCV 2019
			\item 3. 6D Object Pose Estimation survey, kickoff
		  \end{itemize}
	\vspace{0.2cm}
	\item \textbf{DiDi AI Lab} \hfill Beijing, China
          \\ Research Intern \hfill Jul 2017 -- Nov 2017
		  \vspace{-0.1cm}
		  \begin{itemize}
		  	  \itemsep0em 
		      \item 1. Improving CTC models' recognition accuracy on multi-source speech datasets using domain adversarial learning; Transfer methods in image domain adaptation into speech scenario, decrease recognition error by 1-2\%
%		      \item 2. Domain Adversarial Transfer Learning
		  \end{itemize}
	\vspace{0.2cm}
	\item \textbf{South China University of Technology} \hfill Guangzhou, China
        \\ Research Assistant \hfill Oct 2015--Oct 2016
	    \vspace{-0.1cm}
		\begin{itemize}
		    \itemsep0em
			\item 1. Propose UIS to perform Social Networks User Recommendation using LDA; My role is perform ablation studies and paper writing
			\item 2. Participate in National innovation and entrepreneurship training program 2016--2017. Investigate in music recommendation based on audio signals. My role is team leader
			\item 3. Participate in CCF Sougou User Profile Prediction competitions. Ranked 20 among 120 teams. My role is team leader
		\end{itemize}
\end{itemize}

\section*{Work Experience}
\begin{itemize}
	\item \href{https://horizon.ai/}{\textbf{Horizon Robotics}} \hfill Beijing, China
	      \\ Algorithm Engineer - Computer Vision \hfill Apr 2018 -- Feb 2019
	    \vspace{-0.1cm}
		\begin{itemize}
		    \itemsep0em
			\item 1. In charge of Horizon's LiDAR perception project. Build full LiDAR 3D Object Detection pipeline from 0 to 1. Including data annotation rules \& tools; Model design \& training; Deployment on FPGA
			\item 2. Work out a real-time LiDAR sensing demo which is presented in CES 2019
		\end{itemize}          
	\item \href{https://www.alibabagroup.com/}{\textbf{Alibaba}} \hfill Hangzhou, China
	      \\ Intern Algorithm Engineer - Machine Learning \hfill Nov 2017 - Mar 2018
%	      \\ Topic: UGC(Video) Recommendation; NLP
	    \vspace{-0.1cm}
		\begin{itemize}
		    \itemsep0em
			\item 1. Use ODPS(a SQL library) to perform large scale user activity analysis and video recommendation.
			\item 2. Feature Engineering using Logistic Regression, GBDT, DNN, LDA to extract features to perform collaborative filtering
		\end{itemize} 
	\item 
\end{itemize}

\vspace{-1cm}


\section*{Publications}
\begin{itemize}
\item \textbf{Benjin Zhu}, Bingqi Ma, ViP: A View Progressive Approach for 3D Object Detection from Point Cloud, 2019, {\it arXiv preprints} 2019.
\item \textbf{Benjin Zhu}, Zhengkai Jiang, Xiangxin Zhou, Zeming Li, Gang Yu, Class-balanced Grouping and Sampling for Point Cloud 3D Object Detection, 2019, {\it arXiv preprints} arXiv:1908.09492.
\item Ke Xu, Xushen Zheng, Yi Cai, Huaqing Min, Zhen Gao, \textbf{Benjin Zhu}, Haoran Xie, Tak-Lam Wong, Improving user recommendation by extracting social topics and interest topics of users in uni-directional social networks, 2018, {\it Knowledge-Based Systems} 140, 120-133.
\end{itemize}

%\section*{Projects}
%\begin{itemize}
%    \item keypoint object detection
%    \item LiDAR perception
%	\item UIS
%	\item DeepAudioMusicRecommendation
%	\item Gitsoo
%\end{itemize}

\section*{Presentations}
\begin{itemize}
	\item Presentation at Workshop on Autonomous Driving held in CVPR 2019
\end{itemize}

\section*{Honors \& Awards}
\begin{tabular}{ll}
    \textbf{2019} & First Place of nuScenes 3D Object Detection challenge in WAD, CVPR \\
    \textbf{2018} & Honor Graduate \\
	\textbf{2015 -- 2017} & SCUT School Scholarship \\
\end{tabular}

\section*{Activities}
\begin{itemize}
	\item Horizon's Career Training at Silicon Valley \hfill Sep 2019
	\item Academic Visit at Carnegie Mellon University \hfill Jan 2017
\end{itemize}

\section*{Skills}
\begin{tabular}{ll}
	\textbf{Programming Languages} & Python, C++, JavaScript \\
	\textbf{Deep Learning Frameworks} & PyTorch, MXNet, TensorFlow, MegDL \\
	\textbf{Backend Frameworks} & MongoDB, MPI, CUDA \\
	\textbf{Languages} & English, Chinese, Japanese \\
\end{tabular}

\end{document}\include{example}